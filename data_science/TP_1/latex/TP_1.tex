\documentclass[a4paper]{article}
%\documentclass[8pt]{report}
%%%%%%%% CREATE DOCUMENT STRUCTURE %%%%%%%%
%% Language and font encodings
\usepackage[english]{babel}
\usepackage[utf8x]{inputenc}
\usepackage[T1]{fontenc}

%\usepackage{subfig}

%% Sets page size and margins
\usepackage[a4paper,top=3cm,bottom=2cm,left=2cm,right=2cm,marginparwidth=1.75cm]{geometry}

%% Useful packages
\usepackage{amsmath}
\usepackage{graphicx}
\usepackage[colorinlistoftodos]{todonotes}
\usepackage[colorlinks=true, allcolors=blue]{hyperref}
%\usepackage{caption}
\usepackage[justification=centering]{caption}
\usepackage{subcaption}
\usepackage{sectsty}
\usepackage{float}
\usepackage{titling} 
\usepackage{blindtext}
\usepackage[square,sort,comma,numbers]{natbib}
\usepackage[colorinlistoftodos]{todonotes}
\usepackage{xcolor}
\usepackage{fancyhdr}
\usepackage{lipsum}

%% definitions 
\definecolor{darkgreen}{rgb}{0.0, 0.4, 0.0}

%% Define your personal info here %%%%%%%%%%%%%%%%%%%%%%%
\newcommand\TPid{1}
\newcommand\TPname{Linear Algebra}
\newcommand\Firstname{Joao Filipe}
\newcommand\Familyname{Costa da Quinta}
\newcommand\Email{Joao.Costa@etu.unige.ch}

%%%%%%%%%%%%%%%%%%%%%%%%%%%%%%%%%%%%%%%%%%%%%%%%%%%%%%%

%%%%%%% Page header %%%%%%
\pagestyle{fancy}
\fancyhf{}
\rhead{TP \TPid: \TPname}
\lhead{\Firstname \Familyname}
\rfoot{Page \thepage}


%%%%%%%% DOCUMENT %%%%%%%%
\begin{document}

%%%% Title Page
\begin{titlepage}

\newcommand{\HRule}{\rule{\linewidth}{0.5mm}} 							% horizontal line and its thickness

\center 
 
% University
\textsc{\LARGE Université de Genève}\\[1cm]

% Document info
\textsc{\Large Data Science}\\[0.2cm]									% Course Code
\HRule \\[0.8cm]
{ \huge \bfseries TP \TPid : \TPname}\\[0.7cm]								% Assignment
\HRule \\[2cm]
\large
\emph{Author:} \Firstname \; \Familyname\\[0.5cm]		
\emph{E-mail:} {\color{blue}\Email}\\[7cm]		
% Author info
% Author info
{\large \today}\\[2cm]
\includegraphics[width=0.4\textwidth]{images/unige_csd.png}\\[1cm] 	% University logo
\vfill 
\end{titlepage}


% ============================================
% ----------------------------------
\newpage
\section*{1 - Matrix}
\subsection*{.1}
When the number of equations, here 3, is strictly larger than the number of variables, here 2, the equations system has no solution.
\subsection*{.2}

% ----------------------------------
\section*{2 - The importance of the mathematical concept behind a code}
\subsection*{.1}
\textbf{$def \ project\_on\_first(u, v)$} receives two column vectors as an argument, and it projects v onto u, the projected vector is usually called v'. Visually, it means that v' and u are collinear. This also means that: $\exists\ \alpha$ tq. $v'*\alpha = u$.
\begin{figure}[H]
\center
\includegraphics[width=0.3\textwidth]{images/projection.PNG}
\caption{Projection of $\vec{v}^{\,}$ onto $\vec{u}^{\,}$ = $\vec{w_{1}}^{\,}$}
\end{figure}
\subsection*{.2}
\textbf{zip()} function takes as argument two python lists of same size. It then merges one value from the first list, with another value from the second list (same index), creating a list of tuples. \\
Let's see an example:
\begin{center}
x = zip([1,2], [3,4]) -> x = [(1,3),(2,4)]
\end{center}
This means that the three last lines of code perform a simple dot operation between the two vectors given as argument to zip().
\begin{center}
It can be rewritten as: r = np.dot(u,v)
\end{center}
\subsection*{.3}
Step 1: find the vector $\vec{w_{2}}^{\,}$ orthogonal to $\vec{u}^{\,}$\\
If we look at Figure 1, we can see that $\vec{w_{1}}^{\,}$ is collinear to $\vec{u}^{\,}$, and that $\vec{w_{2}}^{\,}$ is orthogonal to $\vec{u}^{\,}$. Moreover, $\vec{v}^{\,}$ = $\vec{w_{1}}^{\,}$ + $\vec{w_{2}}^{\,}$, which means we can easily compute $\vec{w_{2}}^{\,}$ if we have already computed $\vec{w_{1}}^{\,}$. 
\begin{center}
$\vec{w_{2}}^{\,}$ = $\vec{v}^{\,}$ - $\vec{w_{1}}^{\,}$
\end{center}
Step 2: Make it so the orthogonal vector $\vec{w_{2}}^{\,}$ has the same norm as vector $\vec{u}^{\,}$\\
We must first compute $||\vec{u}^{\,}||$ as well as $||\vec{w_{2}}^{\,}||$.\\
By multiplying $\vec{w_{2}}^{\,}$ by a given real value $\alpha$ we can find a new vector $\vec{w_{2}'}^{\,}$ that is collinear to $\vec{w_{2}}^{\,}$, but of different norm.
\begin{center}
$\alpha$ = $||\vec{u}^{\,}||$ / $||\vec{w_{2}}^{\,}||$
\end{center}
\textbf{$def\ orthogonal\_norm\_on\_first(u, v)$} is the function inside $some\_script.py$ that does this computation.

% ----------------------------------
\section*{3 - Computing Eigenvalues, Eigenvectors, and Determinants}
\subsection*{.1}
\subsection*{.2}
\subsection*{.3}

% ----------------------------------
\section*{4 - Computing Projection Onto a Line}
\subsection*{.1}
\subsection*{.2}



\end{document}
